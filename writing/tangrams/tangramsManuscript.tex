\documentclass[12pt, floatsintext, jou]{apa6}
\usepackage{amssymb}
\usepackage{graphicx}
\usepackage[outdir=./]{epstopdf}
%\DeclareGraphicsExtensions{.eps}

\usepackage{mathtools}
\usepackage{enumerate}
\usepackage{apacite}
\usepackage{listings}
\usepackage{multirow}
\usepackage{todonotes}
\usepackage{svg}

\newcommand{\den}[2][]{
\(
\left\llbracket\;\text{#2}\;\right\rrbracket^{#1}
\)
}

\newenvironment{figurehere}
	{\def\@captype{figure}}
	{}

\usepackage{lipsum}
%\pagenumbering{gobble}
%\usepackage{apacite}

\linespread{1}
\usepackage{textcomp}
\usepackage{lingmacros}

\DeclareGraphicsRule{.tif}{png}{.png}{`convert #1 `dirname #1`/`basename #1 .tif`.png}
\graphicspath{{./figures/}}
 
 \definecolor{Green}{RGB}{10,200,100}
\newcommand{\ndg}[1]{\textcolor{Green}{[ndg: #1]}}  


\makeatother

\title{Arbitrariness and path dependence in a communication game}
\shorttitle{Tangrams}
\author{Robert X.D. Hawkins, Nicole Maslan, Noah D. Goodman}
\affiliation{Stanford University} 


\abstract{}

\keywords{pragmatics, language, convention, social cognition}

\authornote{This report is based in part on work presented at the 37th Conference of the Cognitive Science Society. The first author is supported by a NSF Graduate Research Fellowship and a Stanford Graduate Fellowship. Correspondence concerning this article should be addressed to Robert X.D. Hawkins, e-mail: rxdh@stanford.edu}

\begin{document}
\maketitle
\section{Introduction}

Successful communication depends on a set of shared linguistic conventions (de Saussure, XX; Lewis, XX). These conventions allow communities of speakers to coordinate group behavior (cite collective behavior lit??), initiate speech acts \cite{Strawson64_IntentionConvention}, and align beliefs or memories \cite{StolkVerhagenToni16_ConceptualAlignment, ComanEtAl16_MnemonicConvergence}. 
%: when someone at the dinner table asks for ``the salt,'' her friends can be confident that she is referring to a shaker containing a condiment and not, say, a furry animal under the table. 
While \emph{global} conventions adopted and sustained throughout a large population of speakers may form over much longer time scales (cite historical ling?), we also effortless coordinate on \emph{ad hoc} or \emph{local} conventions -- or conceptual pacts -- within the span of a single dialogue (cite Herb). For instance, when discussing a complex proposal for a new experiment, a student and advisor may begin to call one level of the manipulation ``condition A,'' which is then mutually understood to refer to the new experiment for the remainder of the conversation. 

A critical aspect of social conventions, and linguistic conventions in particular, is their arbitrariness: ``cat'' (English), ``chat'' (French), and ``m\={a}o'' (Mandarin) are all equally good ways of referring to a feline in their respective language communities. However, recent studies  emphasizing specific ways in which language is non-arbitrary \cite{MonaghanEtAl14_ArbitraryLanguage, DingemanseEtAl15_IconicityLanguage, LewisFrank16_LengthOfWordsComplexity} forces us to consider exactly in what sense arbitrariness is an essential property of conventions. Are all conventions created equal?

We investigate these questions in a classic reference game paradigm using tangrams as stimuli (cite Herb).



\section{Methods}

\subsection{Participants} 

X participants were recruited from Amazon's Mechanical Turk and dynamically paired into dyads. 

\section{Results}

\subsection{Words over time}

\subsection{Pointwise mutual information}

\subsection{Parts of speech}

\subsection{Entropy}

\section{Discussion}



\bibliography{tangrams}
\bibliographystyle{apacite}


\end{document}  
